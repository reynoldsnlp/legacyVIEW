\documentclass[a4paper,twocolumn]{scrartcl}

\usepackage[english]{babel}
\usepackage[utf8x]{inputenc}
\usepackage{aleks}
\usepackage{qtree}
\usepackage{linguex}

\author{Aleksandar Dimitrov}
\title{Rebuilding WERTi: Providing a Platform for Aiding Second Language Acquisition}

\begin{document}

\maketitle

In the course of this presentation I want to provide insight into the
development process of my re-implementation of WERTi, a system to aid second
language acquisition of adult learners. The original implementation of WERTi was
capable of requesting news articles from a certain news web
site\footnote{\texttt{http://www.reuters.com}} and perform different kinds of
input enhancements on it.

While the original implementation was written in the programming language
\emph{Python}, this new design is rooted in \emph{Java} technologies and makes
use of several frameworks:
\begin{itemize}
  \item UIMA - IBM's \emph{Unstructured Information Management Architecture},
    used to process web contents, find natural
    language and annotate it according to its linguistic features.
  \item Java Servlets and Server Pages - for hosting the web content and
    executing server side processing.
  \item GWT - The \emph{Google Web Toolkit}. This technology compiles native
    \emph{Java} code to \emph{JavaScript} which is then executed on the client
    side. It also eases the process of integrating Remote Procedure Calls to
    communicate between server and client side.
\end{itemize}

The new \emph{Java} implementation now supports arbitrary web sites as inputs,
multiple languages and has some new features regarding input enhancement.
Overall the development focus rested on providing a flexible platform that can be
easily extended to include more functionality. UIMA is used on the server side
to provide a flexible analysis framework and GWT, on the client side, provides
an easy way to enrich content of web pages.

\end{document}
