\documentclass{beamer}

\mode<beamer>{
\usetheme{Singapore}
%Berlin
%\usecolortheme{rose}
}

\usepackage[english]{babel}
\usepackage{qtree}
\usepackage{ucs}
\usepackage[utf8x]{inputenc}
\usepackage{amsfonts,amsmath}
\usepackage{linguex}
\usepackage{color}
\title{Rebuilding WERTi: Providing a Platform for Aiding Second Language Acquisition}
\institute{Universität Tübingen}
\author{Aleksandar L. Dimitrov}

\begin{document}

\begin{frame}
  \titlepage
\end{frame}

\begin{frame}
  \frametitle{Outline}
  \tableofcontents
\end{frame}

\section{What is WERTi?}

\subsection{Second Language Acquisition}
\begin{frame}
  \frametitle{Computational Methods in Second Language Acquisition}
  Make use of existing methods in computational linguistics and natural language
  processing\pause
  \begin{itemize}
    \item Increase the learner's awareness of grammatical patterns\ldots{}\pause
    \item \ldots{} trough input enhancement\pause\emph{ (Receptive Presentation) }\pause
    \item \ldots{} interaction with text in the target language\pause
      \begin{itemize}
	\item \emph{Productive Presentation/Perceiving Interaction}\pause
	\item \emph{Controlled Practice/Productive Interaction}\pause
      \end{itemize}
  \end{itemize}
  Leave room for new ideas.\pause
\end{frame}

\subsection{WERTi aids Second Language Acquisition}
\begin{frame}
  \frametitle{WERTi aids Second Language Acquisition}
  WERTi \ldots
  \begin{itemize}
    \item \ldots{} lets the user decide what content they want to see\pause\\
      $\to$ Would a restriction on the input texts be commendable?\pause
    \item \ldots{} provides example implementations of \emph{Receptive Presentation},
      \emph{Productive Presentation} and \emph{Controlled Practice}\pause
    \item \ldots{} is interactive and easily accessible to the user\pause
    \item \ldots{} has been re-written with flexibility in mind, in order to be
      extended by programmers\pause
  \end{itemize}
\end{frame}

\section{Working on WERTi}
\begin{frame}
  \frametitle{Working on WERTi}
  \begin{itemize}
    \item Complete rewrite, using new technologies\pause
    \item Acquisition of technologies proved a great challenge\pause
    \item Weekly design discussions greatly improved the overall designs
    \item Development by one person\pause
      \begin{itemize}
	\item Advantage: greater consistency, rapid prototyping\pause
	\item Disadvantages: fewer resources/capacity, possibility of creating a ``code bomb''\pause
      \end{itemize}
  \end{itemize}
\end{frame}
\subsection{The Goals}
\begin{frame}
  \frametitle{The Goals}
  \begin{itemize}
    \item Make linguistic processing more flexible\pause\\
      $\to$ using UIMA, the Unstructured Information Management Architecture\pause
    \item Provide server $\leftrightarrow$ client side interaction\pause\\
      $\to$ using RPC calls and GWT, the Google Web Toolkit\pause
    \item Extend possible targets to include (possibly) arbitrary web
      content\pause\\
      $\to$ Using heuristics and eventually linguistic methods to extract
      text\pause
    \item Overall:\pause\\
      $\to$ Provide an easy to extend web based platform to aid second language
      acquisition of adult learners
  \end{itemize}
\end{frame}

\section{Architecture}
\begin{frame}
  \frametitle{Architecture}
  WERTi is now modular:
  \begin{itemize}
    \item Two client side code modules,\pause
      \begin{itemize}
	\item one for the application interface\pause
	\item one for the enhancements interface\pause
    \item Server side code and\pause
    \item with in the server side code, linguistic analysis components\pause
  \end{itemize}
\end{frame}
\begin{frame}
  \frametitle{The general Picture}
  (Picture)
\end{frame}
\begin{frame}
  \frametitle{Textual and Linguistic Analysis}
  \begin{itemize}
    \item HTML tag recognition\pause
    \item Classification into relevant chunks of input\pause
    \item Tokenization\pause
    \item Sentence Boundary detection and coherence analysis\pause
    \item Part of Speech Tagging using different taggers\pause
    \item Enhancement Annotations\pause
  \end{itemize}
\end{frame}

\section{Conclusion/The Current Status} \subsection{Feature Proposals}
\begin{frame}
  \frametitle{The Current Status}
  \begin{itemize}
    \item WERTi is largely usable with a few bugs in the processing code (mostly
      HTML sanitization)\pause
    \item The user interface is not yet functional, but we have basic
      functionality\pause
    \item linguistic processing is so far feature complete, stable and
      fast\pause
    \item WERTi needs stress tests with many users to prove its
      scalability\pause
  \end{itemize}
\end{frame}
\begin{frame}
  \frametitle{Feature Proposals}
  \begin{itemize}
    \item A passivator\pause
    \item Partial translation (when ``mousing over'' a word)\pause
    \item Lemmatization could provide fill-in-the-blanks for finite forms\pause
    \item Phrase structures could be parsed and annotated\pause
    \item \ldots{}
  \end{itemize}
\end{frame}
\end{document}
